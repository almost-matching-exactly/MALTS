\section*{Appendix C}
In this section, we discuss our implementation of existing causal inference methods like genmatch, propensity score matching, BART, causal forest, difference of random forest,  prognostic score matching and FLAME. In Section~\ref{sec:Experiments}, we compare the performance of each of these methods with \malts.

We used MatchIt's implementation of genmatch and propensity score matching as it is commonly used by empiricists \citep{matchit2011}. We allowed matching with replacement for creating match groups and estimating CATEs. As MatchIt returns only match groups and CATE estimates for treated units (and not control units), then in order to estimate CATEs for control units, we flipped the sign of the treatment indicators and estimated negative CATEs (we have to estimate negative CATE in this case because we flipped the sign of the treatment indicator, CATE estimates became negative CATE estimates). We merged the CATE estimates for the treated units and control units to get the CATE estimates for every unit in the dataset.

We used the causal forest algorithm as implemented in the `grf' package in R. The settings for causal forest were set to the default designed by the `grf' developer with number of trees equal to $2000$ and $\sqrt{p}+20$ variables tried for each split. 

We performed the same 5-fold CATE estimation procedure for causal forest, analogous to the one used for estimating CATEs using MALTS. We estimate CATEs for both the treated and control units in each estimation set. 

We used Vincent Dorie's R implementation of BART \citep{dbart}. We performed the same 5-fold CATE estimation using BART that we used for MALTS. For each of the $\eta$ folds, we trained two BART models, one for the learning the response function for estimating the potential outcome under control and the other response function for estimating potential outcome under treatment using the training set. The CATEs were estimated by taking the difference of estimated response functions of treated and control units in the estimation set. We also implement a 5-fold FLAME CATE estimation procedure analogous to the one used by MALTS.

Lastly, we implemented 5-fold prognostic score matching using a random forest approach to model the prognostic score function. We fit a model for control units and a model for treated units using the data in the training set. To estimate the CATE for a treated unit in the estimation set, we found k-nearest neighbors in the control set with a similar estimated prognostic score. Analogously, we estimated the CATEs for the control units in the estimation set using the k-nearest treated units with similarity measured using the prognostic score.
