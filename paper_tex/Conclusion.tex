% !TEX root = main.tex
\section{Conclusion and Discussion}\label{sec:Conclusion}
This paper introduces the MALTS algorithm, which learns a distance-metric on the covariate space for use with matching. The learned metric stretches important covariates and compresses irrelevant covariates for outcome prediction in order to produce high-quality matches. 
%MALTS is able to deal with irrelevant covariates by downweighting their importance in the weighted nearest neighbor algorithm that produces the matches. 
Unlike black-box machine learning methods, MALTS produces interpretable matched groups and returns the stretch matrix on covariates for counterfactual prediction. The stretch matrix is chosen here to be diagonal, so that it can be represented using only a few ``stretch'' numbers that determine the importance of each covariate in determining the matched groups. The matched groups arising from these stretch matrices and are thus interpretable.

A natural extension that we are pursuing is to use neural networks or support vector machines to learn a flexible distance metric in a latent space, thus allowing us to match on medical records, images, and text documents.  This will allow us to incorporate complex data structures by introducing a flexible learning framework (e.g., neural networks) for coding the data. That is, we can redefine the distance metric via
\begin{eqnarray*}\nonumber
\textrm{distance}_{\mathcal{M}} (\x_i,\x_j) &=& \langle\phi_{\mathcal{M}}(\x_i),\phi_{\mathcal{M}}(\x_j) \rangle \;\;\;\textrm{or}\\\nonumber
\textrm{distance}_{\mathcal{M}} (\x_i,\x_j) &=& \left(\phi_{\mathcal{M}}(\x_i)-\phi_{\mathcal{M}}(\x_j)\right)^2,
\end{eqnarray*}
where $\phi$ is a summary of relevant data features learned using a complex modeling framework. As deep neural networks mainly show improvements over other methods for problems that do not have natural data representations (computer vision, speech, etc.), we conjecture that the stretch/almost-exact match combination should suffice for most datasets. The MALTS framework can be further extended to deal with missing covariates, and can be adapted to instrumental variables. 
